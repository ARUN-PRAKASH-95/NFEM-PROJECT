\documentclass[a4paper,12pt]{article}
    \usepackage{amsmath}
    \usepackage{graphicx}
    \usepackage{multirow}
    \title{\textbf{Non Linear Finite Element Analysis}}
   
    \author{Prakash Chinnasamy, Arun Prakash Ganapathy}
   
    \usepackage[top=1.3in, bottom=1.1in, left=1.3in, right=1.3in]{geometry}



\begin{document}
\maketitle


\newpage
\section*{Problem Description}
\indent \indent  The problem provided contains a spherical inclusion of radius $r_i$ which undergoes a phase transformation inside an ideally elastic-plastic matrix material. The matrix material is modeled as concentric sphere of radius $r_o$. The phase transformation of the inclusion only leads only to the volumetric strain without change of shape. From the model it is observed that, it exhibits spherical symmetry with respect to the center of the inclusion.\\
\\
 The non-trivial equilibrium condition for the spherical coordinate system r-$\phi$-$\theta$ is given by 
\begin{equation}
0 = \frac{\partial (r^2\sigma_{rr})}{\partial r} - (\sigma_{\phi\phi}+\sigma_{\theta\theta})
\label{strongform}
\end{equation}
The weak form of Eq(1) reads 
\begin{equation}
0 = \delta W = \int_{r_i}^{r_o} \delta\epsilon^T . \sigma r^2 dr - [r^2\sigma_{rr}\delta u_r]_{r=r_i}^{r_o}
\label{weakform}
\end{equation}
The stress components $\sigma_{rr}$,$\sigma_{\phi\phi}$ and $\sigma_{\theta\theta}$ are non-vanishing and the displacement in radial direction $u_r(r)$ is the only non-vanishing displacement component. Since the problem exhibits spherical symmetry, strains in each axis is connected to each other by the radial displacement $u_r$ and the relation reads 

\begin{equation}
\epsilon = \begin{bmatrix}
\epsilon_{rr} = \frac{\partial u_r }{\partial r}\\
\epsilon_{\phi\phi} = \frac{u_r}{r}\\
\epsilon_{\theta\theta} = \frac{u_r}{r}
\end{bmatrix}
\label{strain matrix}
\end{equation}


The boundary conditions for this problem are $\sigma_{rr}$(r=$r_o$) = 0 and $u_r$(r=$r_i$)= $\frac{1}{3}\tau\epsilon_{v}r_{i}$ 
\section*{Implemented theory}
In Finite element method the governing equation used to solve for displacements in static condition is given by 
$$  
G=F_{int} - F_{ext} = 0
$$
From the given weak form(eq(2)), the required parameters to solve the problem in finite element method has been derived. The term that corresponds to the internal force is $$ \int_{r_i}^{r_o} \underline{\delta}\epsilon^T .\underline{\sigma}  r^2 dr$$
 From this term, the parameters internal force($F_{int}$) and stiffness matrix can be derived. The internal force equation reads
 $$F_{int}= \int_{r_i}^{r_o} \underline{B}^T . \underline{\sigma} r^2 dr$$ by applying gauss quadrature with single gauss point we get 
$$  
F_{int} = 2.\underline{B}^T.\underline{\sigma}(\frac{r_1+r_2}{2})^2.J
$$
and the equation that computes stiffness matrix for each element reads $$K_e = \int_{r_i}^{r_o} \underline{B}^T.\underline{C}.\underline{B}.r^2 dr $$ by applying gauss quadrature we get 
$$  
K_e = 2.\underline{B}^T.\underline{C}.\underline{B}.(\frac{r_1+r_2}{2})^2.J
$$
where J is the Jacobian.\\
The strain-displacement relation (B matrix) can be computed from equation(3) using the relation $ \epsilon = [B].\underline{u^e} $. The derivation of [B] matrix follows
$$  
\epsilon = \begin{bmatrix}
\frac{\partial u_r }{\partial r}\\
\frac{u_r}{r}\\
\frac{u_r}{r}
\end{bmatrix}
$$
$$  
\epsilon = \begin{bmatrix}
\frac{\partial }{\partial r}\\
\frac{1}{r}\\
\frac{1}{r}\\
\end{bmatrix}. N .\underline{u}
$$
where N is linear shape function matrix for 1D element and it is given by $$ [N](\xi) = \begin{bmatrix}
\frac{1}{2}(1-\xi),& \frac{1}{2}(1+\xi)
\end{bmatrix}^T $$
From the above relations the [B]matrix can be obtained and it is given as 
$$  
[B] = \begin{bmatrix}
-\frac{1}{2}\frac{\partial \xi}{\partial r}&\frac{1}{2}\frac{\partial \xi}{\partial r}\\

\frac{1-\xi}{2(N_1r_1+N_2r_2)}&\frac{1+\xi}{2(N_1r_1+N_2r_2)}\\
\frac{1-\xi}{2(N_1r_1+N_2r_2)}&\frac{1+\xi}{2(N_1r_1+N_2r_2)}
\end{bmatrix}
$$
by applying gauss quadrature we get
$$  
[B] = \begin{bmatrix}
-\frac{1}{2}\frac{\partial \xi}{\partial r}&\frac{1}{2}\frac{\partial \xi}{\partial r}\\
\frac{1-\xi}{(r_1+r_2)}&\frac{1+\xi}{(r_1+r_2)}\\
\frac{1-\xi}{(r_1+r_2)}&\frac{1+\xi}{(r_1+r_2)}
\end{bmatrix}
$$
where $\frac{\partial \xi}{\partial r}$ is inverse of Jacobian and $r_1$  and $r_2$ are position of each nodes.
\subsection*{Newton-Raphson Scheme}
The Newton-Raphson schme is used to solve the non-linear system of equations approximately. The general form of Newton-Raphson scheme is given by
$$  
 x_{n+1} = x_n - \frac{f(x_n)}{f'(x_n)}
$$
\\
In FEM, the Newton-Raphson scheme is used to solve the non linear equations to find displacements. So in our case, the newton-raphson scheme reads
$$  
[u_{n+1}] = [u_n] - \frac{[G]}{\frac{\partial [G]}{\partial u}}
$$
Since $ [G] = [K].[u] - [F_{ext}] $ , the partial derivative of [G] with respect to [u] ($\frac{\partial G}{\partial u}$) becomes K(stiffness matrix). Therefore the final equation reads
$$  
[u_{n+1}] = [u_n] - [K^{-1}].[G]
$$
\section*{Program Structure}
\indent   \indent With the required parameters derived, the finite element method can now be programmed. The program structure is shown in the following flowchart
\begin{figure}[htbp]

  \includegraphics[width=0.99\textwidth]{{"NFEM Flowchart (1)".jpeg}}
  \caption{Structure of the program}
 \end{figure}
aThe flow of the program is described below
\begin{itemize}
\item The necessary python packages are imported
\item The loadstep is defined in order to analyze the behaviour of the model incrementally
\item Material parameters and boundary conditions are defined
\item Mesh is generated as per the given code snippet
\item For a given loadstep the program enters the Newton-Raphson scheme
\item Inside the Newton-Raphson scheme for each element the element routine and material routine are processed
\item The element routine computes the stiffness matrix and internal force for each element. In order to compute these C matrix and Internal stress are needed, this is performed by the material routine
\item The workflow of the material routine is given in the following flowchart
\begin{figure}[htbp]
\includegraphics[width=0.97\textwidth]{{"Materialroutine".jpeg}}
  \caption{Material routine}
 \end{figure}
\item It is given that  the material is ideally elastic-plastic and that plasticity initiates at $\epsilon_v = (1+\nu)\frac{\sigma_o}{E} $ which is around 13\% of total strain $\epsilon_v$ 0.01. Therefore in the material routine th estrain is calculated and the equivalent von-mises stress is calculated and checked against the yield stress. If the element has gone into  plastic regime, the new tangent stiffnes matrix and the stress are calculated
\item The material routine returns the algorthimically consistent material tangent stiffness matrix and the current stress 
\item The element routine now computes the stiffness matrix and internal force for each element 
\item From the elemental matrices the global matrices is assembeled using the assignment matrix
\item The governing equation can now be solved using the Newton-Raphson scheme and the error can be calculated
\item This error is checked against the convergence criteria if the criteria is met the program proceeds to the next load step
\item The above process is repeated until the whole load is applied 
\item The final results are the plotted and saved in the working directory
\end{itemize}
\section*{Verification}
\indent \indent A load within the elastic regime was applied and a convergence study with respect to number of elements was performed. The results are displayed in Fig3. The correlation of results to the exact solution is also shown in the Fig3. As we move away from the inclusion the displacement and compressive stress should decrease and we infer from the graph that the plots follow the expected trend. It was also verified in the program if the Newton-Raphson method converges within a single iteration. \\ \indent We can also see that ,there is a slight variation between the exact and calculated solution. This can be attributed to the fact that the numerical method can approximate the exact solution but there is alwways some error in computuations. In our particular case the ratio $\frac{r_o}{r_i}$ influences the correlation between the two solutions. We can see from Fig that if the ratio $\frac{r_o}{r_i} $ tends to $\infty$ ($r_o=100,r_i=1$) the graphs almost overlap
\begin{figure}[htbp]
\includegraphics[width=0.95\textwidth]{{"Elastic2".png}}
  \caption{Convergence study and Comparision between Exact and FEM solution in the Elastic range}
 \end{figure}
\begin{figure}[htbp]
\includegraphics[width=0.95\textwidth]{{"Elastic".png}}
  \caption{Convergence study and Comparision between Exact and FEM solution in the Elastic range}
 \end{figure}
\\
\indent In the first step,a study between Loadstep and number of elements is done. The maximum number of elements that a given lodstep can handle and converge to a solution was found by trial and error. The results are shown in Fig4 . The number elements can be chosen from this graph for a given step size
\begin{figure}[htbp]
\includegraphics[width=0.95\textwidth]{{"Trend".png}}
  \caption{Convergence study and Comparision between Exact and FEM solution in the Elastic range}
 \end{figure}
\begin{figure}[htbp]
\includegraphics[width=0.95\textwidth]{{"Result3b_a".png}}
  \caption{Convergence study and Comparision between Exact and FEM solution in the Elastic range}
 \end{figure}
\\\indent In the next step, a convergence study with respect to number of elements was performed with loadstep arbitrarily chosen as 0.01 through trial and error to  obtain a reasonable runtime. The results are shown in the Fig5. It is found that the number of elements between 20-50 are required for the plot to be a good curve.\\
\indent In the next step, a convergence study with respect to loadstep was performed keeping the number of elements 30 based on the previous study. The curves are plotted for loadstep 0.1,0.01,0.001. The results are shown in Fig6. It is observed that a neat curve can be obtained for timesteps from 0.01 and below. While we can also use loadstep 0.001, the computational cost are high and the variation in results between loadstep 0.01 and 0.001 is insignificant. Therefore the $\Delta\tau$  required is finalised as 0.01
\begin{figure}[htbp]
\includegraphics[width=0.95\textwidth]{{"Result3b_b".png}}
  \caption{Convergence study and Comparision between Exact and FEM solution in the Elastic range}
 \end{figure}
\section*{Results}
For the final loadstep and number of elements chosen the results are plotted for $\tau=1$ and the results are shown in the Fig7. The radial stress evolution of the innermost element throughout the entire loading is shown in Fig7d.\\
The displacement $u_r$ is plotted as a polar graph in Fig8 to visualize the radial displacement through the matrix material
\begin{figure}[htbp]
\includegraphics[width=0.95\textwidth]{{"Result4".png}}
 \caption{Convergence study and Comparision between Exact and FEM solution in the Elastic range}
 \end{figure}
\begin{figure}[htbp]
\includegraphics[width=0.95\textwidth]{{"Polardisplacement".png}}
 \caption{Convergence study and Comparision between Exact and FEM solution in the Elastic range}
 \end{figure}
\end{document}
